\documentclass[12pt,a4paper]{article}

% ============================================================
% PACKAGES ESSENTIELS
% ============================================================
\usepackage[utf8]{inputenc}
\usepackage[T1]{fontenc}
\usepackage[french]{babel}
\usepackage[margin=2.5cm]{geometry}
\usepackage{amsmath}
\usepackage{graphicx}
\usepackage{hyperref}
\usepackage{array}

% ============================================================
% CONFIGURATION
% ============================================================
\hypersetup{
    colorlinks=true,
    linkcolor=blue,
    citecolor=blue,
    urlcolor=blue,
}

% Espacement entre les paragraphes
\setlength{\parskip}{0.5em}

% ============================================================
% DÉBUT DU DOCUMENT
% ============================================================
\begin{document}

% ============================================================
% PAGE DE TITRE
% ============================================================
\begin{titlepage}
    \centering
    \vspace*{2cm}
    
    {\Huge\bfseries Plateforme MicroLearn\par}
    \vspace{1cm}
    {\Large Service de Sélection Intelligente de Modèles\par}
    \vspace{0.5cm}
    {\large Micro2 - ModelSelector\par}
    
    \vspace{2cm}
    
    \includegraphics[width=0.3\textwidth]{example-image} % Remplacez par votre logo
    
    \vfill
    
    {\large Rapport Technique\par}
    \vspace{0.5cm}
    {\large \today\par}
\end{titlepage}

% ============================================================
% TABLE DES MATIÈRES
% ============================================================
\tableofcontents
\newpage

% ============================================================
% CONTENU PRINCIPAL
% ============================================================

\section{Service de Sélection Intelligente de Modèles (Micro2)}

\subsection{Présentation Générale}

Le \textbf{Service de Sélection Intelligente de Modèles} (Micro2 - ModelSelector) est un composant essentiel de la plateforme MicroLearn. Il agit comme un conseiller intelligent qui recommande automatiquement les meilleurs algorithmes d'apprentissage automatique pour les données fournies.

Ce service intervient après la préparation des données (Micro1) et avant l'entraînement des modèles (Micro3). Il constitue le pont intelligent entre les données nettoyées et les algorithmes qui apprendront à partir de celles-ci.

\subsection{Problématique et Objectifs}

\subsubsection{Problématique Adressée}

Dans le monde de l'apprentissage automatique, l'une des difficultés majeures consiste à choisir le bon algorithme parmi des dizaines de possibilités. Cette décision nécessite habituellement :

\begin{itemize}
    \item Une connaissance approfondie de chaque algorithme
    \item Une compréhension fine des caractéristiques des données
    \item Une expérience pratique pour anticiper les performances
    \item Du temps pour tester différentes approches
\end{itemize}

\textbf{Le problème} : La majorité des utilisateurs ne possèdent pas cette expertise, ce qui crée une barrière à l'entrée importante.

\subsubsection{Objectifs du Service}

\begin{enumerate}
    \item \textbf{Démocratiser l'accès à l'IA} : Permettre à tous d'utiliser l'apprentissage automatique sans formation spécialisée
    \item \textbf{Gagner du temps} : Éliminer les phases d'expérimentation aléatoire
    \item \textbf{Optimiser les résultats} : Proposer les algorithmes ayant le plus de chances de réussite
    \item \textbf{Réduire les coûts} : Éviter les essais-erreurs coûteux en ressources de calcul
\end{enumerate}

\subsection{Fonctionnement du Service}

Le service fonctionne selon un processus en trois étapes simples :

\begin{center}
\textbf{Entrée} $\rightarrow$ \textbf{Analyse} $\rightarrow$ \textbf{Sortie}

Données $\rightarrow$ Étude automatique $\rightarrow$ Recommandations
\end{center}

Le service est organisé autour de quatre composants principaux :

\begin{enumerate}
    \item \textbf{Réception des Données} : Accepte les fichiers CSV ou récupère les données depuis le stockage
    \item \textbf{Analyseur de Données} : Examine en profondeur les caractéristiques des données
    \item \textbf{Catalogue de Modèles} : Base de connaissances contenant plus de 20 algorithmes
    \item \textbf{Moteur de Sélection} : Intelligence qui fait correspondre les données avec les meilleurs modèles
\end{enumerate}

\subsection{Fonctionnalités Principales}

\subsubsection{Analyse Automatique des Données}

Le service examine automatiquement le jeu de données pour comprendre :

\paragraph{Caractéristiques Structurelles}
\begin{itemize}
    \item \textbf{Taille} : Nombre de lignes et de colonnes
    \item \textbf{Types de données} : Numériques, textuelles, dates
    \item \textbf{Volume} : Petit dataset ($<$ 1000 lignes) ou grand dataset ($>$ 100 000 lignes)
\end{itemize}

\paragraph{Type de Problème}
Le système détermine automatiquement si le problème relève de :
\begin{itemize}
    \item \textbf{Classification} : Prédire une catégorie (ex: chat ou chien, spam ou non-spam)
    \item \textbf{Régression} : Prédire une valeur numérique (ex: prix, température)
    \item \textbf{Clustering} : Regrouper des éléments similaires sans catégories prédéfinies
\end{itemize}

\paragraph{Qualité des Données}
\begin{itemize}
    \item Détection des valeurs manquantes
    \item Identification des déséquilibres dans les données
    \item Évaluation de la complexité du problème
\end{itemize}

\paragraph{Variable Cible}
Identification automatique de la colonne à prédire, basée sur :
\begin{itemize}
    \item Le nom de la colonne (détection de mots-clés)
    \item Les caractéristiques statistiques
    \item La position dans le fichier
\end{itemize}

\subsubsection{Recommandations Intelligentes}

Pour chaque dataset analysé, le service fournit :

\paragraph{Liste de Modèles Recommandés}
\begin{itemize}
    \item \textbf{Top 5 des meilleurs candidats} classés par pertinence
    \item \textbf{Score de compatibilité} pour chaque modèle (0-100\%)
    \item \textbf{Justification détaillée} expliquant pourquoi ce modèle convient
\end{itemize}

\paragraph{Informations par Modèle}
\begin{itemize}
    \item \textbf{Nom et description} en langage simple
    \item \textbf{Points forts} : Quand utiliser ce modèle
    \item \textbf{Limitations} : Situations où il est moins adapté
    \item \textbf{Niveau de complexité} : Simple, intermédiaire ou avancé
    \item \textbf{Interprétabilité} : Facilité à comprendre les résultats
\end{itemize}

\subsection{Catalogue de Modèles}

Le service dispose d'un catalogue riche de \textbf{plus de 20 algorithmes} d'apprentissage automatique, organisés en catégories distinctes :

\subsubsection{Modèles à Base d'Arbres}

\textit{Utilisation} : Polyvalents, adaptés à la majorité des problèmes

\begin{itemize}
    \item \textbf{Random Forest} : Ensemble d'arbres de décision travaillant en collaboration
    \begin{itemize}
        \item \textit{Idéal pour} : Données mixtes, problèmes complexes
        \item \textit{Avantage} : Très robuste, peu sensible aux erreurs
    \end{itemize}
    
    \item \textbf{Decision Tree} : Arbre de décision simple
    \begin{itemize}
        \item \textit{Idéal pour} : Débutants, visualisation des décisions
        \item \textit{Avantage} : Facile à comprendre et expliquer
    \end{itemize}
\end{itemize}

\subsubsection{Modèles d'Ensemble Avancés}

\textit{Utilisation} : Performances maximales sur données structurées

\begin{itemize}
    \item \textbf{XGBoost} : Algorithme de gradient boosting optimisé
    \begin{itemize}
        \item \textit{Idéal pour} : Compétitions, performance optimale
        \item \textit{Avantage} : Souvent le plus performant
    \end{itemize}
    
    \item \textbf{LightGBM} : Version rapide du gradient boosting
    \begin{itemize}
        \item \textit{Idéal pour} : Grands datasets, rapidité d'entraînement
        \item \textit{Avantage} : Très rapide, efficace en mémoire
    \end{itemize}
\end{itemize}

\subsubsection{Modèles Linéaires}

\textit{Utilisation} : Données avec relations simples et linéaires

\begin{itemize}
    \item \textbf{Régression Logistique} : Classification binaire ou multi-classe
    \item \textbf{Régression Linéaire} : Prédiction de valeurs continues
    \item \textbf{Ridge \& Lasso} : Versions régularisées de la régression linéaire
\end{itemize}

\subsubsection{Modèles de Proximité}

\textit{Utilisation} : Données où les éléments similaires sont proches

\begin{itemize}
    \item \textbf{K-Nearest Neighbors (KNN)} : Classification par voisinage
    \begin{itemize}
        \item \textit{Idéal pour} : Petits datasets, données bien groupées
        \item \textit{Avantage} : Intuitif, pas de phase d'entraînement
    \end{itemize}
\end{itemize}

\subsubsection{Modèles Probabilistes}

\textit{Utilisation} : Données textuelles et catégorielles

\begin{itemize}
    \item \textbf{Naive Bayes} : Classification probabiliste
    \begin{itemize}
        \item \textit{Idéal pour} : Texte, détection de spam
        \item \textit{Avantage} : Très rapide, efficace sur le texte
    \end{itemize}
\end{itemize}

\subsubsection{Support Vector Machines}

\textit{Utilisation} : Séparation complexe de classes

\begin{itemize}
    \item \textbf{SVC \& SVR} : Machines à vecteurs de support
    \begin{itemize}
        \item \textit{Idéal pour} : Problèmes non-linéaires de taille moyenne
        \item \textit{Avantage} : Efficace en haute dimension
    \end{itemize}
\end{itemize}

\subsubsection{Réseaux de Neurones}

\textit{Utilisation} : Problèmes très complexes nécessitant de la puissance

\begin{itemize}
    \item \textbf{MLP (Multi-Layer Perceptron)} : Réseau de neurones classique
    \begin{itemize}
        \item \textit{Idéal pour} : Données complexes, relations non-linéaires
        \item \textit{Avantage} : Flexible, peut apprendre des patterns complexes
    \end{itemize}
\end{itemize}

\subsubsection{Modèles de Clustering}

\textit{Utilisation} : Découverte de groupes naturels dans les données

\begin{itemize}
    \item \textbf{K-Means} : Regroupement en clusters sphériques
    \item \textbf{DBSCAN} : Clustering basé sur la densité
    \item \textbf{Hierarchical Clustering} : Clustering hiérarchique
\end{itemize}

\subsection{Processus de Sélection}

\subsubsection{Étape 1 : Analyse Initiale}

Le système commence par examiner le dataset :

\begin{enumerate}
    \item \textbf{Lecture des données} : Import et validation du fichier
    \item \textbf{Exploration statistique} : Calcul des métriques de base
    \item \textbf{Détection du type de problème} : Classification, régression ou clustering
    \item \textbf{Identification de la cible} : Quelle variable cherche-t-on à prédire ?
\end{enumerate}

\subsubsection{Étape 2 : Filtrage des Candidats}

Parmi les 20+ modèles disponibles, le système élimine ceux qui ne conviennent pas :

\textbf{Critères d'élimination} :
\begin{itemize}
    \item Modèles incompatibles avec le type de tâche
    \item Modèles nécessitant plus de données que disponible
    \item Modèles trop complexes pour la taille du dataset
    \item Réseaux de neurones si non demandés explicitement
\end{itemize}

\subsubsection{Étape 3 : Calcul des Scores de Compatibilité}

Pour chaque modèle restant, un score de 0 à 100 est calculé en évaluant :

\paragraph{Adéquation avec la Taille des Données (30\% du score)}
\begin{itemize}
    \item Petits datasets $\rightarrow$ Modèles simples favorisés
    \item Grands datasets $\rightarrow$ Modèles complexes acceptables
    \item Pénalité si le modèle nécessite plus de données que disponible
\end{itemize}

\paragraph{Compatibilité avec les Types de Données (25\% du score)}
\begin{itemize}
    \item Données numériques uniquement $\rightarrow$ Avantage aux modèles linéaires et SVM
    \item Mélange numérique/catégoriel $\rightarrow$ Avantage aux arbres
    \item Beaucoup de catégories $\rightarrow$ Avantage aux arbres et ensembles
\end{itemize}

\paragraph{Performance et Efficacité (20\% du score)}
\begin{itemize}
    \item Équilibre entre précision attendue et temps d'entraînement
    \item Prise en compte de la métrique d'optimisation choisie
\end{itemize}

\paragraph{Robustesse aux Problèmes Courants (15\% du score)}
\begin{itemize}
    \item Gestion des valeurs manquantes
    \item Résistance au déséquilibre des classes
    \item Sensibilité aux valeurs aberrantes
\end{itemize}

\paragraph{Interprétabilité (10\% du score)}
\begin{itemize}
    \item Bonus pour les modèles faciles à expliquer si demandé
    \item Importance dans les domaines régulés (finance, santé)
\end{itemize}

\subsubsection{Étape 4 : Classement et Recommandation}

Les modèles sont triés par score décroissant et les 5 meilleurs sont retournés avec :

\begin{itemize}
    \item \textbf{Rang} : Position dans le classement
    \item \textbf{Score final} : Pourcentage de compatibilité
    \item \textbf{Justification} : Explication des points forts pour ce dataset
    \item \textbf{Suggestions de paramètres} : Configuration initiale recommandée
    \item \textbf{Warnings} : Limitations potentielles à connaître
\end{itemize}

\subsection{Critères de Décision}

\subsubsection{Critères Principaux}

\paragraph{Taille du Dataset}

\begin{table}[h]
\centering
\begin{tabular}{|l|l|p{6cm}|}
\hline
\textbf{Taille} & \textbf{Nombre de Lignes} & \textbf{Modèles Recommandés} \\
\hline
Très Petit & $<$ 100 & Modèles simples (Régression linéaire, KNN) \\
\hline
Petit & 100 - 1 000 & Arbres simples, modèles linéaires \\
\hline
Moyen & 1 000 - 10 000 & Random Forest, XGBoost, SVM \\
\hline
Grand & 10 000 - 100 000 & Ensembles avancés, réseaux de neurones \\
\hline
Très Grand & $>$ 100 000 & LightGBM, réseaux de neurones \\
\hline
\end{tabular}
\caption{Recommandations selon la taille du dataset}
\end{table}

\paragraph{Type de Tâche}

\textbf{Classification} :
\begin{itemize}
    \item Classes équilibrées $\rightarrow$ Tous modèles de classification
    \item Classes déséquilibrées $\rightarrow$ Random Forest, XGBoost
    \item Classification binaire $\rightarrow$ Régression logistique ou Random Forest
    \item Multi-classes $\rightarrow$ XGBoost, Random Forest, réseaux de neurones
\end{itemize}

\textbf{Régression} :
\begin{itemize}
    \item Relations linéaires $\rightarrow$ Régression linéaire, Ridge, Lasso
    \item Relations complexes $\rightarrow$ Random Forest, XGBoost, MLP
    \item Outliers présents $\rightarrow$ Random Forest (robuste)
\end{itemize}

\textbf{Clustering} :
\begin{itemize}
    \item Groupes bien séparés $\rightarrow$ K-Means
    \item Formes complexes $\rightarrow$ DBSCAN
    \item Besoin de hiérarchie $\rightarrow$ Hierarchical Clustering
\end{itemize}

\subsection{Cas d'Usage}

\subsubsection{Cas 1 : Prédiction de Churn Client}

\textbf{Contexte} : Une entreprise souhaite prédire quels clients risquent de partir

\textbf{Données} :
\begin{itemize}
    \item 50 000 lignes de clients
    \item 15 colonnes (âge, ancienneté, utilisation, support contacté, etc.)
    \item Variable cible : "a\_quitté" (Oui/Non)
\end{itemize}

\textbf{Analyse par le Service} :
\begin{itemize}
    \item Type détecté : Classification binaire
    \item Taille : Grande (50 000 lignes)
    \item Types : Mélange numérique/catégoriel
    \item Classes : Déséquilibrées (90\% restent, 10\% partent)
\end{itemize}

\textbf{Recommandations} :
\begin{enumerate}
    \item \textbf{XGBoost} (Score: 94\%) - Gère parfaitement le déséquilibre
    \item \textbf{Random Forest} (Score: 89\%) - Robuste et interprétable
    \item \textbf{LightGBM} (Score: 87\%) - Très rapide
    \item \textbf{Régression Logistique} (Score: 72\%) - Simple, bon baseline
    \item \textbf{MLP} (Score: 68\%) - Si complexité nécessaire
\end{enumerate}

\textbf{Choix optimal} : XGBoost pour maximiser la détection des départs

\subsubsection{Cas 2 : Estimation de Prix Immobilier}

\textbf{Contexte} : Prédire le prix d'un bien immobilier

\textbf{Données} :
\begin{itemize}
    \item 5 000 annonces
    \item 12 variables (surface, pièces, quartier, année construction, etc.)
    \item Variable cible : "prix" (numérique)
\end{itemize}

\textbf{Analyse par le Service} :
\begin{itemize}
    \item Type détecté : Régression
    \item Taille : Moyenne (5 000 lignes)
    \item Types : Majoritairement numérique
    \item Relations : Partiellement linéaires
\end{itemize}

\textbf{Recommandations} :
\begin{enumerate}
    \item \textbf{Random Forest Regressor} (Score: 91\%) - Capture non-linéarités
    \item \textbf{XGBoost Regressor} (Score: 88\%) - Performance optimale
    \item \textbf{Ridge Regression} (Score: 76\%) - Interprétable
    \item \textbf{Lasso Regression} (Score: 75\%) - Sélection de features
    \item \textbf{SVR} (Score: 71\%) - Gère bien la complexité
\end{enumerate}

\textbf{Choix optimal} : Random Forest pour équilibre performance/interprétabilité

\subsubsection{Cas 3 : Segmentation de Clientèle}

\textbf{Contexte} : Identifier des groupes de clients similaires pour marketing ciblé

\textbf{Données} :
\begin{itemize}
    \item 20 000 clients
    \item 8 variables comportementales (fréquence achat, panier moyen, etc.)
    \item Pas de variable cible (apprentissage non supervisé)
\end{itemize}

\textbf{Analyse par le Service} :
\begin{itemize}
    \item Type détecté : Clustering
    \item Taille : Grande (20 000 lignes)
    \item Types : Numérique uniquement
    \item Structure : Groupes probables
\end{itemize}

\textbf{Recommandations} :
\begin{enumerate}
    \item \textbf{K-Means} (Score: 93\%) - Rapide, efficace
    \item \textbf{DBSCAN} (Score: 78\%) - Détecte formes complexes
    \item \textbf{Hierarchical} (Score: 71\%) - Visualisation de la hiérarchie
\end{enumerate}

\textbf{Choix optimal} : K-Means pour segmentation claire en 4-5 segments

\subsection{Avantages et Bénéfices}

\subsubsection{Pour les Utilisateurs Non-Experts}

\textbf{Accessibilité} :
\begin{itemize}
    \item Pas besoin de connaître les algorithmes en détail
    \item Interface simple : fournir les données, recevoir des recommandations
    \item Explications en langage clair, pas de jargon technique
\end{itemize}

\textbf{Confiance} :
\begin{itemize}
    \item Recommandations justifiées et expliquées
    \item Transparence sur les critères de sélection
    \item Possibilité de comprendre pourquoi tel modèle est proposé
\end{itemize}

\subsubsection{Pour les Data Scientists}

\textbf{Gain de Temps} :
\begin{itemize}
    \item Élimination de la phase d'exploration manuelle des modèles
    \item Point de départ intelligent pour l'expérimentation
    \item Focus sur l'optimisation plutôt que la découverte
\end{itemize}

\textbf{Baseline Automatique} :
\begin{itemize}
    \item Obtention rapide d'une référence de performance
    \item Comparaison systématique avec les meilleurs standards
    \item Documentation automatique des essais
\end{itemize}

\subsubsection{Pour l'Organisation}

\textbf{Réduction des Coûts} :
\begin{itemize}
    \item Moins de temps d'expert nécessaire par projet
    \item Évite les calculs coûteux d'essais-erreurs
    \item Accélération du time-to-market
\end{itemize}

\textbf{Qualité et Cohérence} :
\begin{itemize}
    \item Méthodologie éprouvée et systématique
    \item Réduction des erreurs de choix d'algorithme
    \item Capitalisation de l'expérience dans le catalogue
\end{itemize}

\subsection{Conclusion}

Le \textbf{Service de Sélection Intelligente de Modèles} constitue une brique fondamentale pour rendre l'intelligence artificielle accessible à tous. En automatisant l'étape critique du choix d'algorithme, il supprime l'une des principales barrières à l'adoption de l'IA.

\subsubsection{Points Clés à Retenir}

\begin{enumerate}
    \item Plus de 20 algorithmes disponibles couvrant tous les types de problèmes courants
    \item Analyse automatique des données pour comprendre leurs caractéristiques
    \item Recommandations intelligentes basées sur des critères objectifs et éprouvés
    \item Explications claires permettant de comprendre les choix proposés
    \item Gain de temps significatif dans la phase de démarrage des projets
\end{enumerate}

\subsubsection{Impact}

Ce service transforme le processus de développement de solutions d'IA en :
\begin{itemize}
    \item Démocratisant l'accès aux technologies d'apprentissage automatique
    \item Accélérant les phases initiales de tout projet de data science
    \item Standardisant les bonnes pratiques de sélection de modèles
    \item Réduisant les risques d'erreur dans le choix d'approche
\end{itemize}

\subsubsection{Perspectives}

Le service s'inscrit dans une vision plus large de plateforme AutoML complète, où l'ensemble du cycle de vie d'un modèle - de la préparation des données au déploiement en production - est automatisé et optimisé.

\textbf{Évolutions futures} :
\begin{itemize}
    \item Enrichissement continu du catalogue avec de nouveaux algorithmes
    \item Amélioration de l'intelligence de sélection avec apprentissage sur les succès passés
    \item Intégration de benchmarks automatiques pour valider les recommandations
    \item Personnalisation selon les préférences et contraintes de l'utilisateur
\end{itemize}

% ============================================================
% FIN DU DOCUMENT
% ============================================================
\end{document}
